%%%%%%%%%%%%%%%%%%%%%%%%%%%%%%%%%%%%%%%%%
% Medium Length Professional CV
% LaTeX Template
% Version 2.0 (8/5/13)
%
% This template has been downloaded from:
% http://www.LaTeXTemplates.com
%
% Original author:
% Trey Hunner (http://www.treyhunner.com/)
%
% Important note:
% This template requires the resume.cls file to be in the same directory as the
% .tex file. The resume.cls file provides the resume style used for structuring the
% document.
%
%%%%%%%%%%%%%%%%%%%%%%%%%%%%%%%%%%%%%%%%%

%----------------------------------------------------------------------------------------
%	PACKAGES AND OTHER DOCUMENT CONFIGURATIONS
%----------------------------------------------------------------------------------------

\documentclass{resume} % Use the custom resume.cls style
\usepackage{hyperref}
\usepackage{graphicx}

\usepackage[left=0.75in,top=0.6in,right=0.75in,bottom=0.6in]{geometry} % Document margins
\newcommand{\tab}[1]{\hspace{.2667\textwidth}\rlap{#1}}
\newcommand{\itab}[1]{\hspace{0em}\rlap{#1}}



\clearpage
\name{Dinindu Thilakarathna} % Your name
\address{\href{http://www.ce.pdn.ac.lk}{Department of Computer Engineering, Faculty of Engineering, University of Peradeniya, Sri Lanka}} % Your address
%\address{123 Pleasant Lane \\ City, State 12345} % Your secondary addess (optional)
\address{\href{mailto:e16366@eng.pdn.ac.lk}{e16366@eng.pdn.ac.lk} \\ \href{https://dinindu.me}{https://dinindu.me} \\ +94777186434 }

\begin{document}

%----------------------------------------------------------------------------------------
%	EDUCATION SECTION
%----------------------------------------------------------------------------------------

\begin{rSection}{}
\begin{rSubsection}{}{}{}{}
A computer Engineering undergraduate with a passion to learn and understand the theory and motivation to work on practical projects and produce results.  Experienced in design, development, and implementation of software and hardware solutions.  I am a self-motivated student who is currently doing well in both academics and extra projects. I have initiated a Startup, TroniCode Engineering (PVT) Ltd, that builds hardware and software solutions for business applications of small to medium scale businesses.  The work I have done was possible because of my passion and the capability of solving engineering problems with my skills and knowledge.  With my leadership, management, and communication skills I can take responsibility and work with minimal supervision.

\end{rSubsection}
\end{rSection}


%----------------------------------------------------------------------------------------
%	Interests
%----------------------------------------------------------------------------------------

\begin{rSection}{Interests}
%\begin{rSubsection}{}{}{}{}
\begin{tabular}{l l l} 
Robotics and Automation & Computer Architecture & Embedded Systems \\
Mobile Application Development & Web Application Development & Machine Learning and AI \\
Algorithmic Problem Solving & 3D Modeling & PCB Designing
\end{tabular}
%\end{rSubsection}


\end{rSection}


%----------------------------------------------------------------------------------------
%	EDUCATION SECTION
%----------------------------------------------------------------------------------------
\begin{rSection}{Education}

{\bf \href{http://eng.pdn.ac.lk}{University of Peradeniya}} \hfill {\em 2017 Nov - Present} 
\\ \href{http://www.ce.pdn.ac.lk/undergraduate-courses}{Undergraduate in Computer Engineering BSc. Engineering(Hons.)} \hfill {\bf  GPA: 3.90/4.00} 
\\ {\em 2nd of 60 Computer Engineering Students in E/16 batch}
%Minor in Linguistics \smallskip \\
%Member of Eta Kappa Nu \\
%Member of Upsilon Pi Epsilon \\

{\bf{Bandarawela Central Collage, Bandarawela}} \hfill {\em 2003 - 2016} 
\\ G.C.E. Advanced Level Examination
\\ District Rank - 10, Island Rank - 471 \hfill {\bf  Z-Score: 2.0702}
\\ Physics (A), Combined Mathematics (A), Chemistry (B)



% \begin{rSubsection}{}{}{}{}
% \end{rSubsection}
\end{rSection}


%----------------------------------------------------------------------------------------
%	COMPANY SECTION
%----------------------------------------------------------------------------------------
\begin{rSection}{Entrepreneurship}

{\bf \href{https://www.facebook.com/tronicode.engineering}{Tronicode Engineering (PVT) Ltd (Hardware Team Lead, CO-Founder)}} \hfill {\em 2020 - Present} 
\\ {\em Mobile application, web development and embedded system development company}

%Minor in Linguistics \smallskip \\
%Member of Eta Kappa Nu \\
%Member of Upsilon Pi Epsilon \\

\\ {\bf Projects}
\begin{itemize}
\item {Gedarata Elavalu App ( on PlayStore )}
\item {Disinfection Chambers Electrical and control system (Diyathalawa base hospital)}
\item {Drone control system project}
\end{itemize}
% \begin{rSubsection}{}{}{}{}
% \end{rSubsection}

\end{rSection}


%----------------------------------------------------------------------------------------
%	TECHNICAL STRENGTHS SECTION
%----------------------------------------------------------------------------------------


\begin{rSection}{Skills}

\begin{tabular}{ @{} >{\bfseries}l @{\hspace{6ex}} l }
Programming Languages &  Python, Java, JavaScript, C, C++, Dart \\
Procedural Programming & ARM Assembly \\
Numerical Computing Packages &  MATLAB, Octave, Numpy \\
Hardware Programming  & Arduino, Verilog HDL \\
Mobile Application  & Flutter, Android Studio \\
Web Application  & Python Django, PHP Laravel \\
3D Modelling & Fusion360, Solid Works, AutoCad\\
PCB Related  & Soldering, PCB design and development\\
% Languages &  English, Sinhala \\

\end{tabular}
\end{rSection}

\clearpage

\begin{rSection}{Projects}
\href{https://dinindu.me#projects}{Project Repositories - https://dinindu.me\#projects} \\ \\
\begin{rSubsection}{Obstacle robots for swarm robotic project}{\em {2020-2021}}{}{}
\item Obstacle bot capable of positioning themselves without colliding with each other for the existing swarm robotic platform. \item 
\textit{Technologies: Python, OpenCV, numpy, MQTT, JavaScript, GRPC}
\item \textit{Techniques: Image Processing, stochastic gradient descent, Encryption }
\end{rSubsection}

\begin{rSubsection}{Crypto currency ticker site}{\em {2020-2021}}{}{}
\item Crypto currency to fiet currency conversion sites with high refresh rates And analysing trends real time \item 
\textit{Technologies: Python, MQTT, JavaScript, CSS, Web Sockets, Nginx, Python Django}
\item \textit{Techniques: Real time web socket connections to clients for fast refresh rates}
\end{rSubsection}

\begin{rSubsection}{Vegetable delivery system seller app (Elavalu Gowiya on playstore)}{\em {2020-2021}}{}{}
\item Vegetable delivery system, Seller registration, Item Management and Order management \item 
\textit{Technologies: Flutter, Firebase, Fire-store, Cloud Functions}
\item \textit{Techniques: Firebase and Firestore based App for Android and IOS}
\end{rSubsection}

\begin{rSubsection}{Line Following robot using a camera feed and Machine Learning }{\em {2020}}{}{}
\item A robot which can follow a line based on the camera feed. I have created a neural network from scratch and trained it with the data set gathered by driving the robot \item 
\textit{Technologies: , Raspberry Pi, Image Processing, Custom build robot}
\item \textit{Techniques: Custom Neural Network, OpenCV }
\end{rSubsection}

\begin{rSubsection}{Design and installation of fully automated system for Disinfection Chambers for Diyathalawa Base Hospital, Bandarawela Hospital }{\em {2020}}{}{}
%\item Atmel microcontrollers, Solenoid valves and Contactors for pump \item 
\textit{Technologies: C++ Atmel Microcontollers, Costom PCB Design}
\item \textit{Techniques: Embedded systems, Micro controller Programming, 220v Pump Controlling using contactors}
\end{rSubsection}

\begin{rSubsection}{Intelligent water tank filling system to Hotel ALOFT Grand, Ella}{\em {2020}}{}{}
\item Intelligent tank filling system, tanks located in three height levels. Controlling water pumping and solenoid valves to direct water to tanks necessary\item 
\textit{Technologies: C++ Atmel Micro controller, Custom PCB design }
\item \textit{Techniques: Embedded systems, MicroController Programming, 220v Pump Controlling using contactors }
\end{rSubsection}

\begin{rSubsection}{Verilog Based CPU}{\em {2020}}{}{}
\item Designing of a 32-bit CPU which supports simple instructions with caching. \item 
\textit{Technologies: Verilog}
\item \textit{Techniques: Computer Architecture }
\end{rSubsection}


\begin{rSubsection}{8-bit Computer}{\em {2020}}{}{}
\item Design and building a 8-bit computer. This was designed in gate level and implemented as physical device\item 
\textit{Technologies: Embedded system, Integrated circuits, Simulation}
\item \textit{Techniques: Computer Architecture }
\end{rSubsection}

\begin{rSubsection}{Escape route generation system during floods}{\em {2019}}{}{}
\item Generates escape routes during floods by estimating water levels in rivers and inundation maps. \item 
\textit{Technologies: Flutter, Google Maps Direction API}
\item \textit{Techniques: Shortest Path and optimization algorithms}
\end{rSubsection}

%\clearpage

%----------------------------------------------------------------------------------------

\begin{rSubsection}{Guidance Assisting System for visually impaired persons}{\em {2019}}{}{}
\item Stick with sensors and chest mount camera to process images, notify the details about surrounding to the blind person through a vibration belt. And a user friendly app with voice control capabilities ( to find the stick if misplaced or notify relatives if the blind person needs help ). 
\item \textit{Technologies: Arduino Microcontroller, Raspberry Pi, Ultrasonic Sensors, Gyroscope, OpenCV, Tensorflow} 
\item \textit{Techniques: Sensor Calibration, Vibration Pattern Optimization, Android Studio, Bluetooth Communication}
\end{rSubsection}

\begin{rSubsection}{Various Other Hobby/Course Projects}{\em {2018-2019}}{}{}
\item Analog Line Follower (PD Controller based), Logic Gate level implementation of a 4-digit pass-code lock, Smart home system with raspberry PI, Wire Bending Machine (CNC)
\item \textit{Technologies: Analog and Digital Electronics, Arduino}
\end{rSubsection}

%	EXAMPLE SECTION
%----------------------------------------------------------------------------------------

\begin{rSection}{Achievements}

{\bf IEEExtreme 14.0} \hfill {\em 2020}
\\ (Country Rank - 9, World Rank - 124 more than 2200 teams)
\\Task : Algorithmic Programming Competition


{\bf Google Code Jam} \hfill {\em 2020}
\\Selected to round 2 (world rank - 3847 more than 96,000 teams)
\\Task : Algorithmic Programming Competition

{\bf Google KickStart} \hfill {\em 2020}
\\Round A (world rank - 2777 more than 100,000 teams)
\\Task : Algorithmic Programming Competition

{\bf UOJ Corders v8.0 } \hfill {\em 2019}
\\Second Runners-up (more than 100)
\\Task : Inter University Algorithmic Programming Competition

{\bf Pre-Extreme } \hfill {\em 2019}
\\Second Runners-up (more than 30 teams)
\\Task : Intra University Algorithmic Programming Competition

{\bf IEEExtreme 13.0} \hfill {\em 2019}
\\ (Country Rank - 7, World Rank - 172 more than 4000 teams)
\\Task : Algorithmic Programming Competition

{\bf ACES Corders v8.0 } \hfill {\em 2019}
\\Country Rank - 5 (more than 120 teams)
\\Task : Inter University Algorithmic Programming Competition

{\bf ACES Hackathon } \hfill {\em 2019}
\\Finalists
\\Project : SAFERO (Safe Route Escape System During Flood)

{\bf SLIIT Overnight Hackathon } \hfill {\em 2019}
\\Finalists
\\Project : Algorithmic Programming Competition

{\bf SLIIT CODEFEST } \hfill {\em 2019}
\\Merit Award
\\Project : You See World (Personal Guidance System to Blind people)

{\bf HACKDEV YOUTH SOCIAL INNOVATION CHALLENGE } \hfill {\em 2019}
\\Selected to final teams (Get an opportunity and funds to develop our project as a product)
\\Project : You See World (Personal Guidance System to Blind people)

{\bf IEEE ADVANCING TECHNOLOGY FOR HUMANITY HUMANITARIAN TECHNOLOGY PRODUCT COMPETITION, GENERAL TRACK } \hfill {\em 2018}
\\Runners UP
\\Project : You See World (Personal Guidance System to Blind people)

{\bf ACES Corders v7.0 } \hfill {\em 2018}
\\Country Rank - 7 (more than 120 teams)
\\Task : Inter University Algorithmic Programming Competition

{\bf IEEE SS12 AGE OF INNOVATION PRODUCT COMPETITION} \hfill {\em 2018}
\\Runners UP
\\Project : You See World (Personal Guidance System to Blind people)

\end{rSection}

% \clearpage


% \begin{rSubsection}{Analogue Line follower}{2018}{}{}
% \item A fully analogue line follower which uses a PD control system without microcontrollers.
% \item \textit{Technologies: Basic Analog Electronics}
% \end{rSubsection}
\end{rSection}

%\clearpage
%----------------------------------------------------------------------------------------
\begin{rSection}{Teaching Experience} \itemsep -3pt
\item Teaching Assistant, University of Peradeniya 
\begin{itemize}
    \item CO321 - Embedded Systems \hfill {\em 2021 - present} 
    \item CO224 - Computer Architecture \hfill {\em 2020 - 2021} 
\end{itemize}
\item Sessions on algorithmic programming, Hackers Club, University of Peradeniya \hfill {\em 2020 - 2021} 

 
\end{rSection}

\begin{rSection}{Extra-Curricular} \itemsep -3pt
\item Active member ACES of the University of Peradeniya \hfill {\em 2020 - preset} 
\item Active member of the Hacker's club of the University of Peradeniya \hfill {\em 2020 - present} 
% \item Active member Science Society, Bandarawela Central College \hfill {\em 2014 - 2016} 

 
\end{rSection}


% \begin{rSection}{Other Interests and Hobbies} \itemsep -3pt
% \item 3D modeling and digital art Enthusiast.
% \item Designing Embedded systems.
% \item Music
% \item Athletics


% \end{rSection}

% \clearpage

% REFERENCES

\begin{rSection}{References}
\itab{\href{}{\textbf{Prof. Roshan G. Ragel} }} \\
% \itab{BSc Eng} \\ 
\itab{Professor, Dept. of Computer Engineering} \\
\itab{Univeristy of Peradeniya} \\  
\itab{\href{mailto:roshanr@eng.pdn.ac.lk}{roshanr@eng.pdn.ac.lk}}\\
%\itab{+94-77-3857755} \\

\itab{\href{}{\textbf{Dr. Isuru Nawinne} }} \\
% \itab{BSc Eng} \\ 
\itab{Senior Lecturer, Dept. of Computer Engineering} \\
\itab{Univeristy of Peradeniya} \\  
\itab{\href{mailto:isurunawinne@eng.pdn.ac.lk}{isurunawinne@eng.pdn.ac.lk}}\\
%\itab{+94-71-8495506} \\

\end{rSection}

\end{document}
